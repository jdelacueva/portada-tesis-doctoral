\documentclass[a4paper,12pt]{book}
\usepackage[utf8]{inputenc}
\usepackage[T1]{fontenc}
\usepackage{graphicx}
\usepackage[english,spanish]{babel}

% Instrucciones:
% Creamos el comando \tituloUCM
% Este comando se define en la cabecera del documento en latex
% y se emplea dentro del documento

\newcommand*{\tituloUCM}{%

  \thispagestyle{empty}
  \begin{center}

    % Cabeceras
    \textsc{\LARGE Universidad Complutense de Madrid}\\[1.0cm] % Nombre de la universidad
    \textsc{\Large Facultad de Filosofía}\\[1.0cm] % Nombre de la facultad
    \textsc{\large Departamento de Filosofía del Derecho, Moral y Política II}\\[1.5cm] % Nombre del departamento

    % Logotipo
    % Sustituir por el de la universidad de que se trate
    % Los logotipos de las universidades los encuentras mediante
    % cualquier búsqueda en Google Images
    \includegraphics[width=40mm]{logoucm.eps}\\[2.0cm]
    
    % Título
    \textsc{Tesis doctoral}\\[0.5cm]
    \textsc{\Large Pragmáticas tecnológicas ciudadanas y regeneración
      democrática}\\[1.5cm] % Título de la tesis
    
    % Doctorando
    \begin{minipage}{0.4\textwidth}
      \begin{flushleft} \large
        \emph{Doctorando:}\\
        Javier \textsc{de la Cueva González-Cotera} % Nombre del doctorando
      \end{flushleft}
    \end{minipage}
    ~
    % Director de la tesis
    \begin{minipage}{0.4\textwidth}
      \begin{flushright} \large
        \emph{Director:} \\
        Dr. Andoni \textsc{Alonso Puelles} % Nombre del director
      \end{flushright}
    \end{minipage}\\[0.5cm]
    
    % Fecha
    {\large Madrid, 2014}\\[0cm] % Fecha
    
    \vfill % Fill the rest of the page with whitespace
  \end{center}
}

% Comenzamos el documento

\begin{document}

% Front matter
%---------------------------------------------------------------------

%\frontmatter

\tituloUCM % <------ Título
%\cleardoublepage % Genera una página par en blanco

% Aquí el demás contenido de la tesis doctoral

%\include{licencia}
%\include{dedicatoria}
%\clearpage
%\tableofcontents*
%\listoftables
%\listoffigures
%\include{english-abstract}
%\include{agradecimientos}
%\include{acronimos}
%\include{cita-inicial}

% Main matter
%---------------------------------------------------------------------

%\mainmatter

%\include{introduccion}
%\include{capitulo1}
%\include{capitulo2}
% ...
%\include{conclusiones}

% Back matter
%---------------------------------------------------------------------

%\backmatter

% Bibliografía
% Indices...

\end{document}

%%% Local Variables: 
%%% mode: latex
%%% TeX-master: t
%%% End: 
